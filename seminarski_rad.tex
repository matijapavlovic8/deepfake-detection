\documentclass[seminarski, times, utf8]{fer}


\usepackage{blindtext}


%--- PODACI O RADU / THESIS INFORMATION ----------------------------------------

% Naslov na engleskom jeziku / Title in English
\title{Deepfake detection using GAN}

% Naslov na hrvatskom jeziku / Title in Croatian
\title{Detekcija uvjerljivog krivotvorenog sadržaja na društvenim mrežama uporabom GAN-ova}

% Autor / Author
\author{Barbara Kos, Matija Pavlović}

\voditelj{prof. dr. sc. Tomislav Hrkać}

% Datum rada na engleskom jeziku / Date in English
\date{}

% Datum rada na hrvatskom jeziku / Date in Croatian
\date{}

%-------------------------------------------------------------------------------


\begin{document}


% Naslovnica se automatski generira / Titlepage is automatically generated
\maketitle


%--- SAŽETAK / ABSTRACT --------------------------------------------------------

% Sažetak na hrvatskom
\begin{sazetak}
  Unesite sažetak na hrvatskom.
\end{sazetak}

\begin{kljucnerijeci}
 
\end{kljucnerijeci}

% Sažetak na hrvatskom
\begin{abstract}
  Abstract
\end{abstract}

\begin{keywords}
 
\end{keywords}


% Sadržaj se automatski generira / Table of contents is automatically generated
\tableofcontents


%--- UVOD / INTRODUCTION -------------------------------------------------------
\chapter{Uvod}
\label{pog:uvod}
\section{Pojava}
Prva poznata pojava pojma "deepfake" datira iz prosinca 2017. godine kada je korisnik Reddita osnovao "subreddit" pod nazivom "r/deepfakes".% citiraj https://www.britannica.com/technology/deepfake
Ovaj podforum uglavnom je sadržavao pornografski sadržaj u kojem su izmjenjena lica kako bi nalikovali poznatim osobama. Ovaj fenomen predstavlja tehnološki napredak, ali istovremeno izaziva zabrinutost zbog potencijalne zloupotrebe.

Takvi sadržaji često su prikazivali poznate i utjecajne osobe u situacijama koje se nikada nisu dogodile. Neki od poznatijih primjera uključuju papu Franju, bivšeg predsjednika SAD-a Donalda Trumpa te druge javne ličnosti. Ovaj trend je brzo stekao popularnost, često zbog senzacionalizma i šoka koji izaziva.

Unatoč negativnom kontekstu u kojem se često spominje stvaranje deepfake sadržaja, važno je napomenuti da postoji i pozitivan aspekt primjene ove tehnologije. Naime, deepfake tehnologija može se koristiti u edukativne svrhe, kao što je stvaranje videa u kojima poznate osobe, poput Davida Beckhama, podižu svijest o globalnim problemima poput malarije na različitim jezicima.

Osim toga, deepfake tehnologija nalazi primjenu u umjetnosti i zabavi, npr. u stvaranju scena u filmovima nakon smrti glumaca ili njihova digitalnog starenja. Također, postoji značajan broj deepfakeova čija je svrha isključivo humoristična, a takvi se sadržaji često viralno šire društvenim mrežama.

Važno je razumjeti da deepfake tehnologija nosi sa sobom i etičke izazove te da njezina primjena zahtijeva odgovornost kako bi se izbjegla potencijalne šteta i zloupotreba.
\section{Opasnosti}


%-------------------------------------------------------------------------------
\chapter{Razrada}
\section{Metode stvaranja}
\section{Artefakti}
\section{GAN-ovi}
GAN-ovi su predstavljeni 2014. godine od strane Ian Goodfellowa i njegovih kolega.
Osnovna ideja iza GAN sustava jest postojanje dva glavna dijela mreže: generator i diskriminator.
Tijekom treninga, generator i diskriminator se natječu jedan protiv drugoga. Generator pokušava poboljšati svoje sposobnosti generiranja tako da vara diskriminator, dok diskriminator nastoji postati sve bolji u razlikovanju pravih podataka od lažnih. Ovaj suparnički proces dovodi do poboljšanja kvalitete generiranih podataka tijekom vremena.
%Generalno o GANu
\subsection{Generator}
Generator GAN-a je ključni dio sustava čija je zadaća stvaranje novih uzoraka ili podataka koji bi trebali biti što je moguće sličniji stvarnim primjerima iz skupa podataka na kojem je mreža trenirana. 
Na ulaz generatora dovodi se nasumični šum, a zatim se on izmjenjuje u izlaz koji nalikuje podatcima iz skupa za treniranje. Uvođenjem nasumičnog šuma, postižemo generiranje raznolikih podataka, uzorkovanjem iz različitih točaka ciljne distribucije.
Generator dakle kreira sadržaj suparničkim pristupom, pokušava prevariti diskriminator.
%zadace Generatora
\subsection{Diskriminator}
%zadace diskriminatora

\section {Primjena GAN-ova za detekciju deepfake-ova}
%Ideja za primjenu dolazi iz činjenice da se upravo GAN-ovi često koriste kako bi se kreirali deepfakeovi i oni sami interno pokušavaju detektirati deepfakeove pomocu svog diskiminatora, ovo je vec pokusao taj i taj i ostvario takve i takve rez

%-------------------------------------------------------------------------------
\chapter{Rezultati i rasprava}
\label{pog:rezultati_i_rasprava}
Rekreirali smo taj i taj rad, ostvarili te i te rezultate, možemo ih koristiti za to i to



%--- ZAKLJUČAK / CONCLUSION ----------------------------------------------------
\chapter{Zaključak}
navesi budući rad, detekcija u videima itd.
\label{pog:zakljucak}



%--- LITERATURA / REFERENCES ---------------------------------------------------

% Literatura se automatski generira / References are automatically generated
% Upiši ime BibTeX datoteke bez .bib nastavka / Enter the name of the BibTeX file without .bib extension
\bibliography{literatura}


%--- PRIVITCI / APPENDIX -------------------------------------------------------
\chapter{Privitci}


\end{document}
